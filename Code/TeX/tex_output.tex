\documentclass[12pt,oneside,a4paper]{scrartcl}
%Einstellungen der Seitenraender
\usepackage[left=1.5cm,right=1.5cm,top=2cm,bottom=2.5cm]{geometry}

\usepackage{ngerman}
\usepackage[utf8]{inputenc}
\usepackage{setspace}
\usepackage[pdftex]{graphicx}

%Zum Erstellen von Diagrammen
\usepackage{tikz}
\usepackage{pgfplots}

\usepackage{bigstrut}

%Hyperlinks innerhalb des PDF Dokuments
\usepackage[bookmarks,
ngerman,
pageanchor,
hyperindex,
hidelinks,
pdffitwindow,
pdftitle={Sudprotokoll: Bierstyle / Biername},
pdfauthor={Stephan Mertens},
]{hyperref}
\usepackage{bookmark}

%um text einzufaerben
\usepackage{color}

%tabular zeug
\usepackage{dcolumn}

% Fuer zusaetzliche Zeichen
\usepackage{textcomp}
\usepackage{wasysym}
\usepackage{marvosym}
\usepackage{siunitx}
\sisetup{
	locale = DE ,
	per-mode = symbol
}
\DeclareSIUnit[number-unit-product = {\,}]\IBU{IBU}
\DeclareSIUnit[number-unit-product = {\,}]\plato{^\circ P}
\DeclareSIUnit[number-unit-product = {\,}]\alfa{\percent\,\alpha}
\DeclareSIUnit[number-unit-product = { }]\Min{Minuten}


% Fuer die Kommentare
\usepackage{changepage}

\begin{document}
%Header
	\begin{minipage}[c]{0.70\textwidth}
		\section*{\hspace{-.4cm}Sudprotokoll: Bierstyle / Biername}
	\end{minipage}
	\begin{minipage}[c]{0.29\textwidth}
		\subsection*{am \today}
	\end{minipage}
	\rule{\textwidth}{1pt}
%
\subsection*{Zutaten}
%
%Schuettung
\paragraph{Schüttung:}
	\begin{tabular}[t]{m{8cm} m{2cm} m{1cm}}
		Malz 1 & \num{666} & \si{\kilogram}\bigstrut\\
		Malz 2 & \num{666} & \si{\kilogram}\bigstrut\\
		Malz 3 & \num{666} & \si{\kilogram} \bigstrut\\ \hline
		Hauptschüttung & \num{666} & \si{\kilogram}\bigstrut\\
		&&\\
		Malz 4 & \num{666} & \si{\kilogram}\bigstrut\\
		Malz 5 & \num{666} & \si{\kilogram}\bigstrut\\
		Malz 6 & \num{666} & \si{\kilogram} \bigstrut\\ \hline
		Zweitschüttung & \num{666} & \si{\kilogram}\bigstrut\\\hline\hline
		Gesamtschüttung & \num{666} & \si{\kilogram}\bigstrut
	\end{tabular}\\

\vspace{.25cm}
\hspace{1cm}Angestrebte Stammwürze: \SI{666}{\plato}
%
%Hopfung
\paragraph{Hopfung:}
	\begin{tabular}[t]{m{2.5cm} m{5cm} m{0.8cm} m{1cm} m{0.8cm} m{1cm}}
		Bitterhopen: & Hopfenname & \num{666} & \si{\gram} & \num{666} & \si{\alfa} \\
		Aromahopfen: & Hopfenname & \num{666} & \si{\gram} & \num{666} & \si{\alfa}
	\end{tabular}\\

\vspace{.25cm}
\hspace{1cm}Errechnete Bittere: \SI{666}{\IBU}
%
%Hefe
\paragraph{Hefe:} 
	Hefename

\pagebreak[3]
%Sudverlauf
\subsection*{Sudverlauf - Infusionsverfahren}	
%
%Maischen
\paragraph{Maischen:}
	\begin{tabbing}\hspace{1cm} \=
		\hspace{1cm} \= \hspace{1cm} \=\hspace{1cm} \=\hspace{1cm} \=\hspace{1cm} \= \hspace{1cm} \= \hspace{1cm} \= \hspace{1cm} \= \hspace{1cm} \= \kill
		\>\SI{666}{\litre} vorlegen bei \SI{666}{\celsius}.\\
		\> \>Aufheizen auf \SI{35}{\celsius}.\\
		\>\textit{Gummi/Glucanaserast} für \SI{666}{\Min} von 666 - 666 Uhr.\\
		\> \>Aufheizen auf \SI{44}{\celsius}.\\
		\>\textit{Ferulasäurerast} für \SI{666}{\Min} von 666 - 666 Uhr.\\
		\> \>Aufheizen auf \SI{52}{\celsius}.\\
		\>\textit{Eiweisrast} für \SI{666}{\Min} von 666 - 666 Uhr.\\
		\> \>Aufheizen auf \SI{64}{\celsius}.\\
		\>\textit{Maltose-/$\beta$-Amylaserast} für \SI{666}{\Min} von 666 - 666 Uhr.\\
		\> \>Aufheizen auf \SI{72}{\celsius}.\\
		\>\textit{Verzuckerungs-/$\alpha$-Amylaserast} für \SI{666}{\Min} von 666 - 666 Uhr.\\
		\> \> \>Jodprobe: \> \> \Square\,positiv \> \> \CheckedBox\,negativ\\
		\> \>Aufheizen auf \SI{78}{\celsius}.\\
		\>\textit{Inaktivierungsrast} für \SI{666}{\Min} von 666 - 666 Uhr.
	\end{tabbing}
	\begin{adjustwidth}{2cm}{0cm}
		\hspace{-1cm}Kommentar: Kommentar zum Maischen
	\end{adjustwidth}
	\begin{center}
		\begin{tikzpicture}
		\begin{axis}[width=1\textwidth,height=0.3\textheight,
		grid=both,
		grid style = dashed,
		xlabel={Zeit (t) / \si{\Min}},
		ylabel={Temperatur ($\vartheta$) / \si{\celsius}},
		xmin=0,		xmax=175,
		ymin=40,	ymax=90,
		legend pos = south east,
		]
		\addplot [mark=none, color=blue] table[x=t, y=theta] {maischeverlauf_soll.csv};
		\addplot [mark=none, color=red] table[x=t, y=theta] {maischeverlauf_ist.csv};
		\legend{soll,ist}
		\end{axis}
		\end{tikzpicture}
		\end{center}

\pagebreak[3]
%Läutern
\paragraph{Läutern:} von 666 - 666 Uhr.
	\begin{tabbing}
		\hspace{1cm} \= \hspace{1cm} \= \hspace{1cm} \= \hspace{1cm} \= \hspace{1cm} \=\hspace{1cm} \=\hspace{1cm} \=\hspace{1cm} \= \kill
		\> \SI{666}{\litre} Wasser vorlegen.\\
		\> \>1. Nachguss \> \> \SI{666}{\litre}\\
		\> \>2. Nachguss \> \> \SI{666}{\litre}\\
		\> \> \>Aufhacken: \> \> \CheckedBox\,ja \> \> \Square\,nein\\
		\> \>2. Nachguss \> \> \SI{666}{\litre}\\\\
		\>Würze: \> \> \> \SI{666}{\plato} bei \SI{666}{\celsius} $\Rightarrow$ \SI{666}{\plato}\\
		\>Glattwasser: \> \> \> \SI{666}{\plato} bei \SI{666}{\celsius} $\Rightarrow$ \SI{666}{\plato}
	\end{tabbing}
	\begin{adjustwidth}{2cm}{0cm}
		\hspace{-1cm}Kommentar: Kommentar zum Läutern
	\end{adjustwidth}

\pagebreak[3]
%Würzekochung
\paragraph{Würzekochung:} für \SI{666}{\Min} von 666 - 666 Uhr.
	\begin{adjustwidth}{2cm}{0cm}
		\hspace{-1cm}Kommentar: Kommentar zum Kochen.
	\end{adjustwidth}

\pagebreak[3]
%Whirpool
\paragraph{Whirpoolrast:} von 666 - 666 Uhr.
	\begin{tabbing}
		\hspace{1cm} \= \hspace{1cm} \= \hspace{1cm} \= \hspace{1cm} \= \hspace{1cm} \= \hspace{1cm} \= \hspace{1cm} \= \hspace{1cm} \= \kill
		\>Ausschlagwürze: \> \> \> \> \>  \SI{666}{\plato} bei \SI{666}{\celsius} $\Rightarrow$ \SI{666}{\plato}\\
		\>Verdünnung: \> \> \> \> \> $(\SI{666}{\litre} \cdot \SI{666}{\plato}) / \SI{666}{\plato} = \SI{666}{\litre}$\\
		\>Sudhausausbeute: \> \> \> \> \> \SI{666}{\percent}\\
		\>Geschätzter Alkoholgehalt: \> \> \> \> \> \SI{666}{\percent}\,Vol.
	\end{tabbing}
	\begin{adjustwidth}{2cm}{0cm}
		\hspace{-1cm}Kommentar: Kommentar zum Whirpool.
	\end{adjustwidth}

\pagebreak[1]
%Hefegabe
\paragraph{Hefegabe} und $O_2$-Gabe um 666 Uhr.
\end{document}
