\documentclass[12pt,oneside,a4paper]{scrartcl}
%Einstellungen der Seitenraender
\usepackage[left=1.5cm,right=1.5cm,top=2cm,bottom=2.5cm]{geometry}

\usepackage{ngerman}
\usepackage[utf8]{inputenc}
\usepackage{setspace}
\usepackage[pdftex]{graphicx}

%Zum Erstellen von Diagrammen
\usepackage{tikz}
\usepackage{pgfplots}

\usepackage{bigstrut}

%Hyperlinks innerhalb des PDF Dokuments
\usepackage[bookmarks,
ngerman,
pageanchor,
hyperindex,
hidelinks,
pdffitwindow,
pdftitle={Sudprotokoll: Ingwer-Ale / Waffelmania},
pdfauthor={Stephan Mertens},
]{hyperref}
\usepackage{bookmark}

%um text einzufaerben
\usepackage{color}

%tabular zeug
\usepackage{dcolumn}

% Fuer zusaaetzliche Zeichen
\usepackage{textcomp}
\usepackage{wasysym}
\usepackage{marvosym}

\begin{document}
%Header
	\begin{minipage}[c]{0.70\textwidth}
		\section*{\hspace{-.4cm}Sudprotokoll: {\color{green}Ingwer-Ale} / {\color{green}Waffelmania}}
	\end{minipage}
	\begin{minipage}[c]{0.29\textwidth}
		\subsection*{am \today}
	\end{minipage}
	\rule{\textwidth}{1pt}
%
\subsection*{Zutaten}
%
%Schuettung
\paragraph{Schüttung:}
	\begin{tabular}[t]{m{8cm} m{2cm} m{1cm}}
		{\color{green}Pilsener Malz} & {\color{green}1,4} & kg\bigstrut\\
		{\color{green}Sauermalz} & {\color{green}0,1} & kg\bigstrut\\
		{\color{green}Haferflocken} & {\color{green}0,05} & kg \bigstrut\\ \hline
		Gesamtschüttung & {\color{red}1,55} & kg\bigstrut
	\end{tabular}\\

\vspace{.25cm}
\hspace{1cm}Angestrebte Stammwürze: {\color{green}9} °P
%
%Hopfung
\paragraph{Hopfung:}
	\begin{tabular}[t]{m{2cm} m{5cm} m{0.5cm} m{1cm} m{0.5cm} m{1cm}}
		Bitterhopen: & {\color{green}Magnum} & {\color{green}10} & g & {\color{green}12,2} & \% $\alpha$ \\
		%Aromahopfen: & & & g & &  \% $\alpha$
	\end{tabular}\\

\vspace{.25cm}
\hspace{1cm}Errechnete Bittere: {\color{red}32} IBU
%
%Hefe
\paragraph{Hefe:}
	{\color{green}Safbrew US-05}
%
%Sudverlauf
\subsection*{Sudverlauf - Infusionsverfahren}	
%
%Maischen
\paragraph{Maischen:}
	\begin{tabbing}\hspace{1cm} \=
		\hspace{1cm} \= \hspace{1cm} \=\hspace{1cm} \=\hspace{1cm} \=\hspace{1cm} \= \hspace{1cm} \= \hspace{1cm} \= \hspace{1cm} \= \hspace{1cm} \= \kill
		\> {\color{green}8}l vorlegen bei {\color{green}64}$^\circ C$.\\
		%\> \> Aufheizen auf 35$^ \circ C$.\\
		%\> \textit{Gummi/Glucanaserast} für Minuten von - Uhr.\\
		%\> \> Aufheizen auf 44$^\circ C$.\\
		%\> \textit{Ferulasäurerast} für Minuten von - Uhr.\\
		%\> \> Aufheizen auf 52$^\circ C$.\\
		%\> \textit{Eiweisrast} für Minuten von - Uhr.\\
		%\> \> Aufheizen auf 64\\
		\> \textit{Maltose-/$\beta$-Amylaserast} für {\color{green}40} Minuten von {\color{red}13:10} - {\color{red}13:55} Uhr.\\
		\> \> Aufheizen auf 72$^\circ C$.\\
		\> \textit{Verzuckerungs-/$\alpha$-Amylaserast} für {\color{green}60} Minuten von {\color{red}14:05} - {\color{red}15:05} Uhr.\\
		\> \> \> Jodprobe: \> \> {\color{green}\Square} positiv \> \> {\color{green}\CheckedBox} negativ\\
		\> \> Aufheizen auf 78$^\circ C$.\\
		\> \textit{Inaktivierungsrast} für {\color{green}10} Minuten von {\color{red}15:10} - {\color{red}15:20} Uhr.\\\\
		\> Kommentar: \>\>\> {\color{green}Etwas grob geschrotet; Haferflocken erst um 14:40 Uhr zugeben}
	\end{tabbing}
	\begin{center}
		\begin{tikzpicture}
		\begin{axis}[width=1\textwidth,height=0.3\textheight,
		grid=both,
		grid style = dashed,
		xlabel={Zeit (t) / Minuten},
		ylabel={Temperatur ($\vartheta$) / $^\circ C$},
		xmin=0,		xmax=175,
		ymin=40,	ymax=100,
		legend pos = north west,
		]
		\addplot [mark=none, color=blue] table[x=t, y=theta] {maischeverlauf_soll.csv};
		\addplot [mark=none, color=red] table[x=t, y=theta] {maischeverlauf_ist.csv};
		\legend{soll,ist}
		\end{axis}
		\end{tikzpicture}
		\end{center}
%
%Läutern
\paragraph{Läutern:} von {\color{red}14:30} - {\color{red}16:10} Uhr.
	\begin{tabbing}
		\hspace{1cm} \= \hspace{1cm} \= \hspace{1cm} \= \hspace{1cm} \= \hspace{1cm} \=\hspace{1cm} \=\hspace{1cm} \=\hspace{1cm} \= \kill
		\> {\color{green}1}l Wasser vorlegen.\\
		\> \> 1. Nachguss \> \> {\color{green}3} l\\
		\> \> 2. Nachguss \> \> {\color{green}2,5} l\\
		\> \> \> Aufhacken: \> \> {\color{green}\CheckedBox} ja \> \> {\color{green}\Square} nein\\\\
		\> Würze: \> \> \> {\color{green}3,5}$^\circ P$ \> bei \> {\color{green}77}$^\circ C$ \> $\Rightarrow$ \> {\color{red}10}$^\circ P$\\
		\> Glattwasser: \> \> \> {\color{green}2}$^\circ P$ \> bei \> {\color{green}48}$^\circ C$ \> $\Rightarrow$ \> {\color{red}4,5}$^\circ P$\\\\
		\> Kommentar: \> \> \> {\color{green}ca. $\frac{3}{4}$l auf den Boden gekippt.}
	\end{tabbing}
%
%Würzekochung
\paragraph{Würzekuchung:} für {\color{green}60} Minuten von {\color{red}16:25} - {\color{red}17:28} Uhr.
	\begin{tabbing}
		\hspace{1cm} \= \hspace{1cm} \= \hspace{1cm} \= \hspace{1cm} \= \hspace{1cm} \= \hspace{1cm} \= \hspace{1cm} \= \hspace{1cm} \= \kill
		\> 100g geriebener Ingwer 15 Minuten vor Kochende zugegeben und bei Kochende entfernt\\\\
		\> Kommentar: \> \> \> {\color{green}kein einziges mal übergekocht}
	\end{tabbing}
%
%Whirpool
\paragraph{Whirpoolrast:} von {\color{red}17:32} - {\color{red}17:55} Uhr.
	\begin{tabbing}
		\hspace{1cm} \= \hspace{1cm} \= \hspace{1cm} \= \hspace{1cm} \= \hspace{1cm} \= \hspace{1cm} \= \hspace{1cm} \= \hspace{1cm} \= \hspace{1cm} \= \hspace{1cm} \= \hspace{1cm} \= \hspace{1cm} \= \kill
		\> 100g geriebener Ingwer in den Whirpool gegeben\\\\
		\> Kommentar: \> \> \> {\color{green}NT hat einige Ingwerstückchen enthalten}\\
		\> \> \> \> {\color{green}NT1l Wasser in die Ausschlagwürze gekippt $\Rightarrow$ 9l}\\
		\> \> Ausschlagwürze:\> \> \> \> \>  {\color{green}5}$^\circ P$ \> bei \> {\color{green}74}$^\circ C$ \> $\Rightarrow$ \> {\color{red}11}$^\circ P$\\
		\> \> Verdünnen: \> \> \> \> \> $({\color{green}9}l \cdot {\color{red}11}^\circ P)/({\color{red}9}^\circ P)={\color{red}11}l$\\
		\> \> Sudhausausbeute: \> \> \> \> \> {\color{red}65} \% mit 11l\\
		\> \> Geschätzter Alkoholgehalt: \> \> \> \> \> {\color{red}3,6}\% Vol.
	\end{tabbing}
%
%Hefegabe
\paragraph{Hefegabe} und $O_2$-Gabe um {\color{green}21:30} Uhr.
\end{document}
