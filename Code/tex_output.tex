\documentclass[12pt,oneside,a4paper]{scrartcl}
%Einstellungen der Seitenraender
\usepackage[left=1.5cm,right=1.5cm,top=2cm,bottom=2.5cm]{geometry}

\usepackage{ngerman}
\usepackage[utf8]{inputenc}
\usepackage{setspace}
\usepackage[pdftex]{graphicx}

%Zum Erstellen von Diagrammen
\usepackage{tikz}
\usepackage{pgfplots}

\usepackage{bigstrut}

%Hyperlinks innerhalb des PDF Dokuments
\usepackage[bookmarks,
ngerman,
pageanchor,
hyperindex,
hidelinks,
pdffitwindow,
pdftitle={Sudprotokoll: Ingwer-Ale / Waffelmania},
pdfauthor={Stephan Mertens},
]{hyperref}
\usepackage{bookmark}

%um text einzufaerben
\usepackage{color}

%tabular zeug
\usepackage{dcolumn}

% Fuer zusaaetzliche Zeichen
\usepackage{textcomp}
\usepackage{wasysym}
\usepackage{marvosym}

\begin{document}
%Header
	\begin{minipage}[c]{0.70\textwidth}
		\section*{\hspace{-.4cm}Sudprotokoll: Ingwer-Ale / Waffelmania}
	\end{minipage}
	\begin{minipage}[c]{0.29\textwidth}
		\subsection*{am \today}
	\end{minipage}
	\rule{\textwidth}{1pt}
%
\subsection*{Zutaten}
%
%Schuettung
\paragraph{Schüttung:}
	\begin{tabular}[t]{m{8cm} m{2cm} m{1cm}}
		Pilsener Malz & 1,4 & kg\bigstrut\\
		Sauermalz & 0,1 & kg\bigstrut\\
		Haferflocken & 0,05 & kg \bigstrut\\ \hline
		Gesamtschüttung & 1,55 & kg\bigstrut
	\end{tabular}\\

\vspace{.25cm}
\hspace{1cm}Angestrebte Stammwürze: 9 °P
%
%Hopfung
\paragraph{Hopfung:}
	\begin{tabular}[t]{m{2cm} m{5cm} m{0.5cm} m{1cm} m{0.5cm} m{1cm}}
		Bitterhopen: & Magnum & 10 & g & 12,2 & \% $\alpha$ \\
		Aromahopfen: & & & g & &  \% $\alpha$
	\end{tabular}\\

\vspace{.25cm}
\hspace{1cm}Errechnete Bittere: 32 IBU
%
%Hefe
\paragraph{Hefe:}
	Safbrew US-05
%
%Sudverlauf
\subsubsection*{Sudverlauf - Infusionsverfahren}	
%
%Maischen
\paragraph{Maischen:}
	\begin{tabbing}
		\hspace{1cm} \= \hspace{1cm} \= \hspace{1cm} \= \hspace{1cm} \= \kill
		\> 8l vorlegen bei 64$^\circ C$.\\
		%\> \> Aufheizen auf 35$^ \circ C$.
		%\> \textit{Gummi/Glucanaserast} für Minuten von - Uhr.
		%\> \> Aufheizen auf 44$^\circ C$.
		%\> \textit{Ferulasäurerast} für Minuten von - Uhr.\\
		%\> \> Aufheizen auf 52$^\circ C$.
		%\> \textit{Eiweisrast} für Minuten von - Uhr.
		%\> \> Aufheizen auf 64
		\> \textit{Maltose-/$\beta$-Amylaserast} für 40 Minuten von 13:10 - 13:55 Uhr.\\
		\> \> Aufheizen auf 72$^\circ C$.\\
		\> \textit{Verzuckerungs-/$\alpha$-Amylaserast} für 60 Minuten von 14:05 - 15:05 Uhr.\\
		\> \> Aufheizen auf 78$^\circ C$.\\
		\> \textit{Inaktivierungsrast} für 10 Minuten von 15:10 - 15:20 Uhr.\\\\
		\> Kommentar: \>\>\> Etwas grob geschrotet; Haferflocken erst um 14:40 Uhr zugeben
	\end{tabbing}
%
\paragraph{Läutern:} von 14:30 - 16:10 Uhr.
	\begin{tabbing}
		\hspace{1cm} \= \hspace{1cm} \= \hspace{1cm} \= \kill
		\>
	\end{tabbing}

\end{document}
